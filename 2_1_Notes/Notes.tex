% \documentclass[12pt]{beamer}
% \usepackage{url}
% % \usepackage[margin=0.8in]{geometry}
% % \usepackage[pdftex]{graphicx}
% \usepackage{subfigure}
% \usepackage[font={footnotesize},labelfont=bf]{caption}
% \usepackage{flushend}
% \usepackage{qtree}
% \usepackage[sort&compress,numbers,super]{natbib}
% % \geometry{a4paper}
%
% \usepackage{amsmath}
% \usepackage{mathrsfs}
% \usepackage{amssymb}
% % \usepackage[T1]{fontenc} % font encoding
% % \usepackage[utf8]{inputenc} % input encoding
% % \usepackage[english]{babel} % keyword translation and hyphenation
% % \usepackage{lmodern} % lmodern looks better than cm-super
%
% \usepackage{wrapfig}
% \usepackage{dblfloatfix}
% \usepackage{textgreek}
% \usepackage{enumitem}
% \usepackage[parfill]{parskip}
% \graphicspath{ {"/Users/nicholasbrady/Google Drive/School/Academic/West Research/Projects/Data Science/PaperFigures/"} }
% \usepackage{makecell}
% \usepackage{setspace}
% % \doublespacing
% \usepackage{placeins}
% %\usepackage[outdir=./]{epstopdf}
% \usepackage{epstopdf}
% \usepackage{authblk}
% % \usepackage{changes}
% \usepackage{color,soul}
% \usepackage{xcolor}




\documentclass[usenames,dvipsnames]{beamer}
\usetheme{Boadilla}
\usefonttheme[onlymath]{serif}
\setbeamercolor{normal text}{fg=white,bg=black}
\setbeamercolor{frametitle}{fg=white}

\usepackage{mathtools}
\usepackage{enumitem}
\usepackage{xcolor}
\usepackage{hyperref}
\hypersetup{colorlinks=true}
\usepackage{outlines}

\setlist[itemize]{label=\textcolor{white}{\textbullet}} % set bullet color white

\newcommand{\flux}{\mathrm{\mathbf{N}}}
\newcommand{\vel}{\mathrm{\mathbf{v}}}
\newcommand{\solcur}{\mathrm{\mathbf{i}}}


\title{\color{white} Nederlands 2.1}
% \subtitle{\color{white} EMI-TFSI with Li-TFSI}

\author{Nicholas Brady}
% \institute{University of Hasselt}
\date{\today}


\begin{document}

	\begin{frame}
		\titlepage
	\end{frame}


	\begin{frame}
		\frametitle{Outline}
		{
			\color{white}
			\hypersetup{linkcolor=white}
			\tableofcontents
		}
	\end{frame}


	\section{November 30}
	\begin{frame}
		\frametitle{Imperatief $\rightarrow$ Je zou ... kunnen (werkwoord)}
		Imperatief\\
		\color{red}{Beleefder: Advies}\\
		Je \underline{zou} ... \underline{kunnen (werkwoord)}.
		\hfill \break
		\normalcolor

		Let op het verbruik als je een nieuw toestel koopt. (letten + op)\\
		\color{red}{Je \underline{zou} op het verbruik \underline{kunnen letten} als je een nieuw toestel koopt.}\\

		\hfill \break
		\normalcolor
		Zet potten op het juiste vuur. (zetten)\\
		\color{red}{Je \underline{zou} potten op het juiste vuur \underline{kunnen zetten}.}\\

		\hfill \break
		\normalcolor
		Koop ledlamp. (kopen)\\
		\color{red}{Je \underline{zou} ledlamp \underline{kunnen kopen.}}\\
	\end{frame}


	\begin{frame}
		\frametitle{Reflexief Werkwoorden}

		\textbf{pagina 113}\\
		Dikketruiendag\\
		De kinderen verkleden zich als schapen.\\

		\hfill \break
		(verkleden zich)
		Mensen verkleden zich als ... \\
		Ik verkleed me als ``nerd." \\

		\hfill \break
		(zich excurseren) Jij excuseert je.\\
		(zich vergissen) Hij/Zij vergist zich.\\
		(zich wassen) Jullie wassen je.\\
		(zich herinneren) We herinneren ons.\\
		(zich scheren) Ze scheren zich.\\

	\end{frame}


	\begin{frame}
		\textbf{pagina 115}\\

		\hfill \break
		afwas machine = vaatwasser

		een strijkijzer: Het is een toestel waarmee je hemden kan strijken.\\
		een frietketel: Het is een toestel waarin je frieten kan bakken.\\
		een thermoskan is een ding waarmee je hete koffie kan houden (besparen).\\

		\hfill \break
		Temperatuur: \\
		\color{red}{heet} $>$ \color{pink}{warm} $>$ \color{white}{lauw} $>$ \color{ProcessBlue}{koud} $>$ \color{blue}{ijskoud}

	\end{frame}


	\begin{frame}
		\frametitle{Trouw Woorden}

		\begin{columns}
		\begin{column}{0.5\textwidth}
			het trouwfeest\\
			het huwelijksfeest\\

			\hfill \break
			de bruid\\
			de bruidegom\\

			\hfill \break
			schminken\\
			het bruidsboeket\\
			het gemeentehuis\\
			de burgemeester\\
			de schepen(en) [assistent burgemeester]\\

		\end{column}
		\begin{column}{0.5\textwidth}  %%<--- here
			de receptie\\
			de kerk\\
			de priester\\

			\hfill \break
		  de trouwfoto's \\
			het trouwfeest (het avondfeest) \\
			de openingsdans \\
			de taart \\
			de huwelijksreis \\
			de trouwring \\
		\end{column}
		\end{columns}

	\end{frame}


	\begin{frame}
		\frametitle{November 30 Boekoefening}

		\begin{itemize}
			\item pagina 113 (video Dikketruiendag)
			\item pagina 115
		\end{itemize}

	\end{frame}


	\section{December 2}
	\begin{frame}

		Toen + bijzin \\
		Toen ik seven jaar was, ging ik naar het basisonderwijs.\\
		\hspace{2.85cm} \ldots\ , leerde ik schaken. \\
		\hspace{2.85cm} \ldots\ , las ik de krant. \\

		\hfill \break
		omdat = want \\
		omdat + bijzin \\
		Ik ben te laat omdat mijn baas mij heeft gebeld. \\
		Ik ben te laat want mijn baas heeft mij gebeld. \break

		je kan met ``omdat" starten, maar je kan niet met ``want" starten \\
		(Omdat mijn baas mij heeft gebeld, ben ik te laat.)

	\end{frame}


	\begin{frame}
		\frametitle{Bijzinnen}

		\begin{center}
			\begin{tabular}{l l}
				\textbf{Bijzinnen} & \textbf{Preposities (+ substantief)} \\
				\hline

				Toen 								& Tijdens \\
				Als (= wanneer) 		& Na \\
				Terwijl 						& Voor \\
				omdat 							& \\
				Nadat 							& \\
				Voordat 						& \\

			\end{tabular}
		\end{center}

		\hfill \break

		\textbf{bijzin:} de werkwoord is in de laatste plaats. \break

		Terwijl hij \underline{fietst}, wast hij zijn kleren. \\
		Nadat ik Nederlands had \underline{gestudeerd}, vond ik een goede job. \\
		Voordat ik een goede job \underline{vond}, had ik Nederlands gestudeerd. \\
	\end{frame}


	\begin{frame}
		\frametitle{Perfectum met Modal Werkwoorden}

		\textbf{kunnen, moeten, willen, mogen, gaan} \break
		\hfill \break

		ik eet $\rightarrow$ ik heb frietjes gegeten. \\
		ik will eten $\rightarrow$ ik heb \ldots\ willen eten. \\
		ik moet eten $\rightarrow$ ik heb \ldots\ moeten eten. \\
		ik ga eten $\rightarrow$ ik heb \ldots\ gaan eten. \\

	\end{frame}


	\begin{frame}
		\frametitle{Boekoefening}

		\begin{itemize}
			\item pagina 167: gebruik het juiste woord (toen, terwijl, en, omdat, want)
			\item pagina 128: Reis naar Pairi Daiza
		\end{itemize}

	\end{frame}

	\section{December 7}
	\begin{frame}
		\frametitle{Imperfectum / Perfectum Oefening}
		pagina 124 \break

		In het huis van de bruid maakte Greet zich klaar. \\
		In het gemeentehuis trouwden de bruid en bruidegom voor de wet. \\
		Ze dansten op het avondfeest. \\

		geven $\rightarrow$ gaven (gaf) \break

		s' ochtends \\
		Ik heb opgestaan om 9 uur. (Ik stond om 9 uur op.) \\
		Eerst, heb ik een douche genomen. (Eerst nam ik een douche.) \\
		Daarna dronk ik witte thee (Jair). \\
		Daarna rookte ik een sigaret. \\
		Dan at ik mijn ontbijt.

	\end{frame}


	\begin{frame}
		\frametitle{Boekoefening}

		\begin{outline}
			\1 pagina 130: Dat moet je echt eens doen
				\2 dat (het woorden)
				\2 die (de woorden)
			\1 (Extra) Een film: Marina
				\2 https://www.imdb.com/title/tt2614860/
		\end{outline}

	\end{frame}

\end{document}
